\documentclass[11pt]{article}
\usepackage[english]{babel}
\usepackage[utf8]{inputenc}
\usepackage{hyperref}
\usepackage{fancybox,graphicx}
\usepackage{subfig}
\usepackage{fancyhdr}
\usepackage[left=3.7cm,top=4cm,right=3.7cm,bottom=4cm]{geometry} 
\usepackage{url}
\usepackage{textcomp}
\usepackage{array}
\usepackage{enumitem}
\usepackage{caption}

\hypersetup{
	plainpages = false, pdfpagelabels,
	bookmarks,
	bookmarksopen = true,
	bookmarksnumbered = true,
	linktocpage,
	% pagebackref,
	colorlinks = true,
	linkcolor = blue,
	urlcolor  = blue,
	citecolor = green,
	anchorcolor = green,
	hyperindex = true,
	hyperfigures,
	pdfauthor={Francisco Javier Pérez Gil},
	pdfcreator={Francisco Javier Pérez Gil},
	pdftitle={Trabajo de Interacción y Visualización de la Información: Tableau}
}

\DeclareFontFamily{\encodingdefault}{\ttdefault}{\hyphenchar\font=`\-}
\graphicspath{{./img/}}

\lhead{}
\chead{\textsc{Scife User Guide}}
\rhead{}
\lfoot{Fco. Javier Pérez Gil \& Raúl Moreno Galdón}
\cfoot{}
\rfoot{\thepage}

\renewcommand{\headrulewidth}{0.5pt} 
\renewcommand{\footrulewidth}{0.4pt} 
\pagestyle{fancy}

\renewcommand{\labelitemi}{$\bullet$}
\renewcommand{\labelitemii}{$\circ$}
\renewcommand{\labelitemiii}{$\triangleright$}

\setlength{\parindent}{0pt}
\begin{document}
\thispagestyle{empty}
\begin{titlepage}
	\begin{center}
		\textsc{\Huge Scife User Guide}
		\vspace*{0.5in}\\
		\huge{\textsc{Universidad de Castilla-La Mancha}}
		\vspace*{0.6in}\\
		\begin{figure}[h!]
			\centering
			\subfloat{
				\includegraphics[scale=0.40]{logo_esii}}
			\hspace{.5cm}
			\subfloat{
				\includegraphics[scale=0.60]{uclm.jpg}}
		\end{figure}
		\vspace{0.6in}
		\huge{\textbf{Authors:}\\Francisco Javier Pérez Gil\\Raúl Moreno Galdón}
	\end{center}
\end{titlepage}
\newpage
\thispagestyle{empty}
\tableofcontents
\clearpage
\thispagestyle{empty}
\listoffigures
\clearpage
\setcounter{page}{1}

\section{Introduction}
This document contains a user manual that explains the user how to use the SCIFE application. Information is structured in different sections, each one contains the instructions of a subject: experiment, applications, and so one.

\section{Experiments}
This section explains the user how to operate with the experiment functionality offered in the SCIFE application.


\end{document}
